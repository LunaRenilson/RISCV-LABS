\documentclass[12pt]{article}

\usepackage[utf8]{inputenc}
\usepackage[portuguese]{babel}
\usepackage{amsmath}
\usepackage{geometry}
\geometry{a4paper, margin=2.5cm}

% Pacotes para código
\usepackage{listings}
\usepackage{xcolor}

% Configuração de estilo para código
\lstdefinestyle{meucodigo}{
  backgroundcolor=\color{gray!10},
  basicstyle=\ttfamily\small,
  frame=single,
  language=C,
  keywordstyle=\color{blue},
  commentstyle=\color{gray},
  stringstyle=\color{red},
  showstringspaces=false
}

\title{Relatório: Inferência de Rede Neural em RISC-V}
\author{Renilson C. de Luna Jr.}
\date{\today}

\begin{document}

\maketitle

\section{Introdução}

\noindent
\hspace{2em}Este relatório descreve a implementação de um processo de inferência de rede neural utilizando arquitetura RISC-V de 32 bits. O objetivo é executar uma rede previamente treinada com pesos quantizados e otimizados, usando operações compatíveis com instruções básicas da ISA RISC-V.

\hspace{2em}O sistema recebe os dados de entrada, executa as camadas da rede (como convoluções, ativações e normalizações) e gera a saída inferida com uso mínimo de instruções e memória. Este trabalho foca em eficiência, portabilidade e clareza do código, destacando os desafios encontrados e as soluções propostas.

\section{Leitura dos Dados (Parser)}

\noindent
\hspace{2em}A leitura dos dados de entrada é uma etapa crucial para o funcionamento da rede neural. O parser é responsável por interpretar os dados brutos e convertê-los em um formato que possa ser utilizado pela rede. A leitura dos pesos da rede foi feita aproveitando o fato de que a quantidade de vetores de uma camada i corresponde ao número de neurônios da camada, e cada vetor contém o número de pesos correspondente ao número de neurônios da camada anterior, segundo o exemplo:

4,10,20,3 \rightarrow 4 camadas:
\begin{itemize}
    \item Entrada: 4 neurônios
    \item Oculta 1: 10 neurônios
    \item Oculta 2: 20 neurônios
    \item Saída: 3 neurônios
\end{itemize}

Logo, as camadas lN tem matriz no formato:

\begin{itemize}
    \item l1: $[10][4]$
    \item l2: $[20][10]$
    \item l3: $[3][20]$
\end{itemize}

\end{document}
